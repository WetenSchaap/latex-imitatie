% This section is meant for a laymens summary in your own language. It is not required at all, but people ternd to include it for family and friends. Sometimes it gets its own chapter, sometimes it is somewhere in the postface. I gave it its own chapter, but that is really up to you.
%%%%%%%%%%%%%%%%%%%%%%%%%%%%%%%%%%%%%%%%%%%%%%%%%%%%%%%%%%%%%%%%%%%%%%%%
% \cleardoubleevenemptypage
\cleardoubleevenemptypage
\chapter{Nederlandstalige samenvatting}
\label{ch:layman-summary}

\pgfmathsetmacro\chapclr{\colourarray[8]}
\hypersetup{
citecolor  = \chapclr,
linkcolor  = \chapclr,
urlcolor   = \chapclr,
}

\epigraph{
  'T sal waerachtig wel gaen
}{
  \textit{Willem Barendsz}
}
%%%%%%%%%%%%%%%%%%%%%%%%%%%%%%%%%%%%%%%%%%%%%%%%%%%%%%%%%%%%%%%%%%%%%%%%
\begin{otherlanguage}{dutch}
 
  Lalalalla Eindelijk schrijf ik eens wat in mijn eigen taal. Hallo papa en mama! Jullie zijn de enige die dit ooit gaan lezen! 

  \lipsum[1-2]

\end{otherlanguage}