% Import packages
\usepackage{etoolbox}          % No clue what this does to be honest.

\usepackage[utf8]{inputenc}    % accept both α and \alpha

\usepackage{graphicx, color}   % For making mice coloured text.

\usepackage{scrhack}           % supress warning. No idea why this works.

\usepackage{textcomp,gensymb}  % Extra symbols, like \celsius.

\usepackage{lipsum}            % For making dummy text
\usepackage{blindtext}         % idem dito

\usepackage[stretch=15]{microtype} % prevents many bad hboxes by allowing stretch of letters by 1.5%
\usepackage{mathtools}         % Tools for maths

\usepackage{siunitx}           % To write nice SI units

\usepackage{listings}          % for showing code nicely

\usepackage[%
  colorlinks = true,
  unicode,
]{hyperref}                    % set internal link colors. Will be set to chapter colors in the mainmatter. The values here are just the fallback and I could just have left them out.

\usepackage{url}               % needed to properly show some links in the bibliography.

\usepackage[perpage,symbol*]{footmisc} % For making footnotes nice (and different from regular references to literature)

\usepackage{epigraph}          % for showing quotes at new chapters

\usepackage{tabularx}          % package for fancy tables

\usepackage{adjustbox}         % package for making tables fit on 1 page

\usepackage[final]{pdfpages}   % for including page-wide images.

\usepackage[%
  cross,
  center,
  width=180mm,
  height=250mm
]{crop}                        % Includes bleed, see the readme. This can be commented out for the reading version of your thesis. The `cross` command just shows you where the paper will be cut, comment it out before sending the pdf to the press. The given values for width and height add 5 mm on all sides and are typically okay.

% These two packages are for cycling trough colours depending on the chapter.
% (optional)
\usepackage{pgffor}
\usepackage{pgfmath}
